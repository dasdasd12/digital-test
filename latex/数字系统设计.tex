\documentclass{template/coloredthesis}
\usepackage{float}
\usepackage{pdfpages}
\usepackage{ctex}
\usepackage{titlesec}
\usepackage{tocloft}
\usepackage{etoolbox}
\usepackage[absolute,overlay]{textpos}
\setlength{\TPHorizModule}{1cm} % 设置水平单位为1cm
\setlength{\TPVertModule}{1cm}  % 设置垂直单位为1cm
\setlength{\parskip}{0.4em}
%————————全篇颜色设置————————
%			c1		c2		c3
\maincolor{305396}{90a9c3}{d8e9e5}

%————————中文注意事项————————
%加粗\textbf 是黑体
%斜体\textit 是楷体

%————————模板注意事项————————
%尤其可修改项:页面布局、设置字体、缩进、中文标题格式设置(“1”还是“一”)、titlepage、封面页中的常量

%彩色表格需要用到以下的包
%\usepackage{booktabs}
%\usepackage{multirow}
%\usepackage{colortbl}
%\usepackage{array}



% 导出时需要注释掉以下两行
% \special{dvipdfmx:config z 0}
% \setdark


% 调整目录中 section 标号与标题的间距,避免重叠
\renewcommand{\cftsecnumwidth}{3em} % 增大 section 标号宽度
\renewcommand{\cftsecindent}{0em}   % section 标号左缩进
\renewcommand{\cftsecleader}{\cftdotfill{1}} % 减少点线间距

% 略微增大目录整体字体,并加粗
\renewcommand{\cfttoctitlefont}{\large\bfseries\centering}
\renewcommand{\cftsecfont}{\normalsize\bfseries}
\renewcommand{\cftsecpagefont}{\normalsize\bfseries}
\renewcommand{\cftsubsecfont}{\normalsize\bfseries}
\renewcommand{\cftsubsecpagefont}{\normalsize\bfseries}
\renewcommand{\cftsubsubsecfont}{\normalsize\bfseries}
\renewcommand{\cftsubsubsecpagefont}{\normalsize\bfseries}
\renewcommand{\cftdotsep}{1} % 减少两个点之间的距离

% 目录也加入目录中
\pretocmd{\tableofcontents}{%
    \addcontentsline{toc}{section}{\contentsname}%
}{}{}

% 减小目录行间距
\makeatletter
\patchcmd{\tableofcontents}{\@starttoc{toc}}{%
        \begingroup
                \let\oldparskip\parskip
                \parskip=0pt
                \let\oldbaselineskip\baselineskip
                \baselineskip=0.59\baselineskip
                \@starttoc{toc}%
                \parskip=\oldparskip
                \baselineskip=\oldbaselineskip
        \endgroup
}{}{}
\makeatother

% 图片编号格式调整为“图1.1”
\renewcommand{\thefigure}{\arabic{section}.\arabic{figure}}
\makeatletter
\@addtoreset{figure}{section}
\makeatother

% 图片编号格式调整为“图1.1”
\renewcommand{\thetable}{\arabic{section}.\arabic{table}}
\makeatletter
\@addtoreset{table}{section}
\makeatother

% \doublespacing % 全局行间距设置为双倍行距
\titleformat{\section}{\Large\bfseries}{\thesection}{1em}{}
\titleformat{\subsection}{\large\bfseries}{\thesubsection}{1em}{}

\begin{document}


\begin{center} % 这是封面

    \vspace{1cm}

    \begin{figure}[!h]
        \centering
        \includegraphics[width=0.8\textwidth]{template/南京理工大学图样.png}
    \end{figure}
    \vspace{2.5cm}

    \fontsize{36}{\baselineskip}{\selectfont \kai 
    \begin{tabular}{c}
        数字系统设计实验报告
    \end{tabular}}

    \vspace{2.5cm}

    \fontsize{23pt}{\baselineskip}\selectfont\kai

    \begin{tabular}{c}
        % &\makebox[4em][c]{实验名称}importan     \hspace{0.2cm}  \dlmu[8cm]{通信电子线路综合实验} \\
        姓\hspace{2em}名 \hspace{0.2cm} \dlmu[8cm]{段蔚伸}       \vspace{0.5cm}   \\
        学\hspace{2em}号 \hspace{0.2cm} \dlmu[8cm]{923132A10309} \vspace{0.5cm}   \\
        专\hspace{2em}业 \hspace{0.2cm} \dlmu[8cm]{微电子科学与工程}   \vspace{0.5cm}   \\
        指导教师         \hspace{0.2cm} \dlmu[8cm]{花汉兵} \vspace{0.5cm} \\
        实验日期         \hspace{0.2cm} \dlmu[8cm]{2025.11.17-2025.11.28}      \vspace{0.5cm}   \\
    \end{tabular}

\end{center}

% \includepdf{template/封面.pdf}

\begin{textblock*}{\textwidth}(3cm,26cm)
    \centering
    \LARGE
    2025.11.26
\end{textblock*}

\thispagestyle{empty}
\newpage

\tableofcontents
\thispagestyle{empty}

\setcounter{page}{1}

\clearpage

\section{实验要求与分析}

本实验要求使用EGO1 FPGA开发板设计一个数字系统,实现DDS信号发生器的基本功能。具体要求包括:
\begin{itemize}
    \item 学号显示:电光23.学号最后4位,例如EO23.0135;微电子23.学号最后4位,例如IC23.0335;将学号显示在EGO1开发板上的数码管;
    \item 若通过开关直接输入频率控制字,计算波形频率的理论值,将理论计算值显示在开发板上的数码管;(若是通过按键设置频率值,则将频率设置值显示在开发板上的数码管)
    \item 基于DDS原理,应用Verilog HDL设计产生正弦信号,用示波器分别观察D/A转换器输出、经滤波后输出的波形 ;
    \item 设计测频电路,将测量的波形频率值显示在EGO1开发板上的数码管;
    \item 设计产生三角波、锯齿波、方波等多种波形;
\end{itemize}
以及进阶要求:
\begin{itemize}
    \item 通过按键设置频率值,频率变化范围1Hz~1MHz;
    \item 测量信号源输出信号的频率,要求信号源输出波形的频率变化范围1Hz~1MHz ,峰峰值0.2V~3V,直流偏置为0V;
    \item 自主发挥,添加其他功能。例如:产生扫频信号、设置信号源直流偏置并测量输出信号的频率… … 。
\end{itemize}
根据要求可以得知,本实验的核心任务是设计一个DDS信号发生器,并实现频率测量功能。除此之外,还需要精确控制数码管的显示内容,以便用户能够直观地看到当前的频率设置和测量结果。从本实验的核心任务出发,可以将设计流程分为以下几个主要模块:
\begin{itemize}
    \item DDS信号发生模块:实现正弦波、三角波、锯齿波和方波的产生;
    \item 频率测量模块:实现对输出信号频率的测量,并将结果显示在数码管上;
    \item 数码管显示模块:实现学号、频率设置值和测量值的显示;
    \item 控制模块:实现按键输入频率设置值的功能。
\end{itemize}
考虑到本实验的难度较低,时间跨度较长,因此针对题目,我设计了自己的解决方案。

首先,对于DDS模块,该模块需要设计ROM存储各种波形的采样数据,这时候需要调取IP核,但是这种设计在移植、修改时会比较麻烦,对于本题目来说,我选择了直接读取初始化文件的方式来实现波形的存储和读取。并且优化了存储深度,只存储了正弦波在第一象限的波形数据,从而节省了存储空间。而且对于方波、三角波和锯齿波,我选择了直接通过逻辑运算、象限判断来实现波形的产生,从而简化了设计。

测频模块我使用了FFT算法来实现频率的测量。通过对输入信号进行采样,并利用FFT算法计算出信号的频率成分,从而得到信号的频率值。该方法对信号质量的要求低,能够测量信噪比较低的信号,且测量速度较快,适合本实验的需求。

控制模块我同时使用了UART和按键两种方式来实现系统的设置。UART方式可以实现上位机控制,方便用户通过串口终端进行设置;按键方式则提供了本地控制的选项,用户可以直接在开发板上进行设置。两种方式结合使用,提高了系统的灵活性和易用性。

最后,数码管显示模块我设计了优雅的显示方案,能够清晰地显示学号、频率设置值和测量值。并结合按键、拨码开关等硬件实现了用户友好的交互界面。

此外,我还设计了扫频功能、电位器调制输出幅度功能等附加功能,进一步提升了系统的实用性和功能丰富性。

\section{模块设计流程}

\subsection{DDS信号发生模块设计}

DDS信号发生器模块作为系统的核心模块,我在设计时着重考虑了资源消耗、可调参数等方面。DDS的基本原理是通过累加相位增量来实现频率控制,然后根据相位值从波形存储器中读取对应的采样值,最后通过D/A转换器输出模拟信号。那么在这个过程中,ROM数据的位宽、相位累加器的位宽和DA输出的位宽都会影响最终输出信号的质量和系统资源的消耗。为了应对多种情况,我设计了多种参数可调的DDS模块,用户可以根据实际需求选择合适的参数组合。这些参数包括:数据位宽、相位位宽、采样点数等。此外,我还设计了多种波形的产生方式,包括正弦波、方波、三角波和锯齿波。对于正弦波,我使用了ROM存储采样数据的方法,而对于其他波形,则通过逻辑运算和象限判断来实现,从而节省了存储资源。

\begin{eqnarray*}
    f_{out} = \frac{\Delta Phase \times f_{clk}}{2^N}
\end{eqnarray*}

\subsection{频率测量模块设计}

\subsection{UART模块设计}

\subsection{数码管显示与控制模块设计}

\subsection{幅度调整功能设计}

\section{总结与体会}

\section{部分核心代码}

\end{document}